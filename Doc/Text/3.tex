\section{Shape Detection}
\label{sec:shape_detection}

\subsection{Basics}

\textbf{Question 1:} Remind the principle of pseudo-inverse approach.

\TODO

\textit{Now you observe a positions of an asteroid at different times, and the objective is to estimate its trajectory.}

\textbf{Question 2:} Complete the program \texttt{'AsteroidTry'} to estimate the asteroid trajectory. Detail the theoretical approach.




\subsection{Shape-Based Approaches}

\textbf{Question 3:} Recall basic principle of shape detection in Computer Vision.

Now, let us consider one planet to be detected and characterised (Fig. 1) in 3 different configurations.

\textbf{Question 4:} Detect \& Characterise (position, size) the 1st planet in Conf. 1.

\textbf{Question 5:} What’s happen with Conf. 2 and Conf. 3?








\subsection{Contour-Based Approaches}

\textbf{Question 6:} Extract the planet (Conf. 1) using N points and next a Pseudo-Inverse Approach. What’s happen in other configurations?

\textbf{Question 7:} Extract the planet (Conf. 1) using N points and next a Pseudo-Inverse Approach. What’s happen in other configurations?

\textbf{Question 8:} Extract the planet (Conf. 1) using an Optimisation algorithm. What’s happen in other configurations?

\textbf{Question 9:} Explain the RANSAC Algorithm. Use it in Conf. 2 and Conf. 3.

\textbf{Question 10:} Compare with the circular Hough Transform.