\section{Introduction to Mathematical Morphology}
\label{sec:introduction_to_mathematical_morphology}



\setcounter{subsection}{1}
\subsection{Erosion and Dilation}
% question 2 image:
\textbf{Question 1} \textit{Define erosion and dilatation from an ensemblist point of view and an functional point of view. Give some properties related to these operators.}

The ensemblist (or set-theoretic) perspective treats an image as a set of foreground pixels. In this context, the erosion ($A \ominus B$) of set A (the image) by a structuring element B results in the set of points where B, translated at each point, is entirely contained within A. This operation shrinks objects and removes small structures.
\[
A \ominus B = \{x \in \mathbb{Z}^n
|
B_x \subseteq A \}
\]
The dilation ($A \oplus B$) of A by B results in the set of points where the reflection of B, translated at each point, intersects A. This operation expands objects and fills small gaps.
\[
A \oplus B = \{x \in \mathbb{Z}^n
|B_x \cap A \neq \emptyset\}
\]

From a functional perspective, images are seen as functions and morphological operations are defined using the maximum and minimum. For erosion:
$(f \ominus B)(x) = \inf_{b \in B} f(x + b)$ this replaces each pixel with the minimum value under the structuring element which makes bright regions shrink. And for dilation: $(f \oplus B)(x) = \sup_{b \in B} f(x - b)$ this replaces each pixel with the maximum value under the structuring element which makes bright regions expand.

Some properties of Erosion and Dilation include:
\begin{enumerate}
    \item \textbf{Duality:} 
    $A \ominus B = (A^c \oplus B)^c$ where \( A^c \) is the complement of set \( A \).

    \item \textbf{Increasing Property:} If \( A \subseteq C \), then: 
    $A \ominus B \subseteq C \ominus B, \quad A \oplus B \subseteq C \oplus B$

    \item \textbf{Extensive and Anti-extensive Properties:} indicate that dilation expands the set, while erosion shrinks it.
    $A \subseteq A \oplus B, \quad A \ominus B \subseteq A$ 

    \item \textbf{Associativity:} Erosion and dilation are associative operations: 
    $(A \oplus B) \oplus C = A \oplus (B \oplus C)$

    \item \textbf{Translation Invariance:} For any translation \( T_t \) (shift by \( t \)), we have: 
    $(T_t A) \ominus B = T_t (A \ominus B), \quad (T_t A) \oplus B = T_t (A \oplus B)$ where \( T_t A = \{ x + t \mid x \in A \} \) is the translation of \( A \).
\end{enumerate}


\newpage
\textbf{Question 2} \textit{Operate in a binary image and a greyscale image with the MatLab commands imerode and imdilate. What are the effects on binary and grayscale images? Justify. Try with different structuring elements (different shapes, different sizes).}

In the figure below, the eroded and dilated images can be seen plotted next to the original image for each structuring element with the following shapes in order: rectangle, disk, diamond and line. \TODO{Mention something about the different quality of output for these? Which one is best? Why is it like the Erosion and Dilation results are flipped? Shouldn't erosion shrink the stuff and dilation expand the stuff?}

\vfill

\begin{figure}[H]
    \centering
    \includegraphics[width=0.75\linewidth]{Doc/Graphics/Part2/part2_Question2.png}
\end{figure}


\newpage
\textbf{Question 3} \textit{Extract internal and external edges of a binary image, and the morphological gradian.}
\begin{figure}[h]
    \centering
    \includegraphics[width=1\linewidth]{Doc/Graphics/Part2/Part2_Question3.png}
\end{figure}


\textbf{Question 4} \textit{As an exercise, write an algorithm that show, in the map of Europe, the distance of each pixel w.r.t. the sea.}
\begin{figure}[h]
    \centering
    \includegraphics[width=0.75\linewidth]{Doc/Graphics/Part2/part2_Question4.png}
\end{figure}



\textbf{Question 5} \textit{Find an algorithm that detect rectangular objects of ’image2.jpg’.}

\begin{figure}[H]
    \centering
    \includegraphics[width=\linewidth]{Doc/Graphics/Part2/Part2_Q5.png}
\end{figure}


\subsection{Morphological Filtering}
\subsubsection{Filters}
\textbf{Question 6} \textit{Define the two morphological filters called opening and closing. What are the effects on a binary image sur as ’image1.jpg’ (use the commands imopen and imclose)?}

\begin{itemize}
    \item \textbf{Opening filter:} is basically erosion followed by dilation. It removes small objects and noise from the foreground and smooths object boundaries.

    \item  \textbf{Closing filter:} is basically dilation followed by erosion. It fills small holes and gaps in the foreground and smooths object boundaries.    
\end{itemize}


The \texttt{imopen()} command seems to reduce some amount of "black-colored noise" from the image and slight sharp inserts in the circle in the bottom left of the original image. The \texttt{imclose()} command seems to remove "white-colored noise" and the outlet of the circle in the top left.

The effects of the commands on \texttt{image1.jpg} can be seen below.

\begin{figure}[H]
    \centering
    \includegraphics[width=0.75\linewidth]{Doc/Graphics/Part2/part2_Q6.png}
\end{figure}

\subsubsection{Form Detection}
\textbf{Question 7} \textit{The objective being to recognize some forms on images, find a simple algorithm to operate form detection}

See Question 5, can easily be modified for different shapes, i.e. circle, diamond etc.


\textbf{Question 8} \textit{Apply a salt-and-pepper noise: what’s happen with your previous algorithm?}

The algorithm breaks and does not function at all due to the noise in the picture. See below:

\begin{figure}[H]
    \centering
    \includegraphics[width=\linewidth]{Doc/Graphics/Part2/part2_Q8.png}
\end{figure}


\subsubsection{Denoising}
\textbf{Question 9} \textit{Use the image Nebuleuse.jpg and apply a salt-and-pepper noise. De-noise the image by filtering.}

\begin{figure}[H]
    \centering
    \includegraphics[width=\linewidth]{Doc/Graphics/Part2/part2_Q9.png}
\end{figure}

\textbf{Question 10} \textit{Apply the same process to the ’Spain Beach’ image to isolate the beach itself.}

\begin{figure}[H]
    \centering
    \includegraphics[width=0.5\linewidth]{Doc/Graphics/Part2/part2_Q10.png}
\end{figure}




\newpage
\subsubsection{Top-Hat \& Black-Hat Filters}
\textbf{Question 11} \textit{Define Top-hat and black-hat process in a 1D function: what is the associated process?}

\begin{itemize}
    \item \textbf{Top-Hat:} This transform is the difference between the original signal and its opening, highlighting small bright features or peaks in the signal.

    \item \textbf{Black-Hat:} This transform is the difference between the closing of the signal and the original signal, highlighting small dark features or valleys.
\end{itemize}


\textbf{Question 12} \textit{Define and operate top-hat and black-hat on a greyscale image. What do you observe?}

\textcolor{red}{TODO Write shit here}


\subsection{Morphological Skeletonization \& Segmentation}
\subsubsection{Skeletonization Process}
\textbf{Question 13} \textit{Write and operate a Skeletonization on the diplodocus.}

\begin{figure}[H]
    \centering
    \includegraphics[width=\linewidth]{Doc/Graphics/Part2/part2_Q13.png}
\end{figure}


\textbf{Question 14} \textit{Based on skelittization, find an algorithm that operate a segmentation in a binary image. Apply on ’image1.jpg’.}

\begin{figure}[H]
    \centering
    \includegraphics[width=\linewidth]{Doc/Graphics/Part2/part2_Q14.png}
\end{figure}



\newpage
\subsubsection{Image Segmentation}
\textbf{Question 15} \textit{Find a Skeletonization algorithm and operate on the Blood Cells image.}

\TODO{Write about why this does not work really well for the "real" image of the blood cells, even if we "binarise" it... Write about what steps we could take to maybe make it work?}

\begin{figure}[H]
    \centering
    \includegraphics[width=\linewidth]{Doc/Graphics/Part2/part2_Q15.png}
\end{figure}
