\section{Introduction to Data Analysis: Data Classification}
\label{sec:data_analysis}

\subsection{Supervised Classification}
\textcolor{blue}{\textbf{Question 1:}}
\textit{Enumerate main supervized classification techniques and describe them in few lines.}

%~~~~~~~~~~~~~~ANSWER~~~~~~~~~~~~~~
Supervised Classification is... ???

Some typical Supervised Classification techniques are listed and described below:
\begin{enumerate}
    \item \textbf{k-Nearest Neighbours (k-NN):} The most simple classification algorithm where the principle is to find the \textit{k} closest data in the dataset to determine its class. 
    First a distance has to be defined, which is not always easy.

    
    \item \textbf{Maximum Likelihood:} A simple classifier based on the probability distribution.
    It assumes the data follows a known probability distribution (typically normal or Gaussian distr.) for each class.
    ML estimates the probability that a given data point belongs to each possible class and assigns it to the class with the highest likelihood.

    \begin{equation*}
        C^* = \underset{C_i}{\operatorname{argmax}} \, P(C_i \mid o)
    \end{equation*}

    
    \item Bayesian Classification:
    Based on the bayesian rule where given an observation \textit{o}, the \textit{prior} (a priori probability), is described as:
    \begin{equation*}
        P(C_i \mid o) = \frac{P(o \mid C_i) P(C_i)}{P(o)}
    \end{equation*}
    where $P(C_i \mid o)$ is called \textit{posterior} (a posterior probability).

    More text??

    
    \item \textbf{Linear Classification:}
    The most simple classification rule, where the objective is to establish a hyperplane, \textit{HP}, that separates the data into two groups:
    \begin{equation*}
    \begin{split}
        &\mathcal{H} : w_0 + \vec{w} \cdot \vec{x} = 0 \\
        &\text{In a \textit{n}-D space:} \\
        &\mathcal{H} : w_0 + w_1 x_1 + w_2 x_2 + \cdots + w_n x_n = 0
    \end{split}
    \end{equation*}

    Linear Classification is only possible if the data is \textit{linearly separable}. If not, there are infinite solutions
    An example of Linear Classification is the Perceptron Algorithm, which bla bla
    
    
    \item Neural Networks:
    \item Decision Trees:
    \item Concept Lattices:
    \item Support Vector Machine
    \item Random Forests:
    \item etc.
\end{enumerate}



\textcolor{blue}{\textbf{Question 2:}}
\textit{Apply and compare the following algorithms:
\begin{enumerate}
    \item 1-NN (your programme)
    \item Neural Network
\end{enumerate}
on 'Spain Beach' image or another of your choice
}


\textcolor{blue}{\textbf{Question 3:}}
\textit{Search on internet one of the two the following algorithms and apply it to the same image
\begin{enumerate}
    \item SVM
    \item Ramdom Forest
\end{enumerate}
}




\subsection{Unsupervised Classification}
\textcolor{blue}{\textbf{Question 4:}}
\textit{Enumerate main unsupervized classification techniques and describe them in few lines.}


\textcolor{blue}{\textbf{Question 5:}}
\textit{Apply the following algorithms with the dedicated objective
\begin{enumerate}
    \item k-means algorithm to define the four classes automatically and speed-up the classification process
    \item Pseudo-Inverse technique to estimate the position of a planet
    \item PCA technique to reduce the size of an image
\end{enumerate}
}